\documentclass[a4paper,12pt]{article}
\usepackage{amsmath, amssymb, graphicx, hyperref}
\usepackage{cite}

\title{Experimental Investigation of Structured Energy Return (SER) Feedback in a Superconducting Qubit-Cavity System}
\author{Ryan Wallace\thanks{\href{https://orcid.org/0009-0009-8725-1876}{ORCID: 0009-0009-8725-1876}} \\ 
\texttt{rathmon@gmail.com}}
\date{\today}

\begin{document}

\maketitle

\begin{abstract}
This experiment investigates quantum work extraction (ergotropy) and coherence retention in a superconducting two-qubit Jaynes-Cummings system with feedback control. By implementing adaptive feedback in a real superconducting circuit, we aim to test whether coherence-based energy retention outperforms entanglement-based strategies in preserving extractable work. Our experimental setup uses a transmon qubit-cavity system in a dilution refrigerator and applies feedback with a tunable delay (\(\tau_f\)) to analyze its impact on system coherence and work extraction.
\end{abstract}

\section{Introduction}
Quantum thermodynamics examines energy conversion in quantum systems, where coherence and entanglement significantly impact energy extraction. The Jaynes-Cummings (JC) model describes the interaction of a two-level system (qubit) with a cavity mode, exhibiting Rabi oscillations. However, in real quantum devices, decoherence rapidly destroys quantum correlations, limiting work extraction.

This experiment seeks to validate the \textbf{Structured Energy Return (SER) hypothesis} in a real superconducting qubit-cavity system. Unlike traditional feedback schemes, SER-based control preserves coherence rather than just suppressing noise. We will investigate:
\begin{itemize}
    \item How feedback delay (\(\tau_f\)) impacts coherence and work retention.
    \item Whether coherence-based feedback improves ergotropy over entanglement-based methods.
    \item The relationship between qubit-cavity coupling (\(g\)), feedback strength (\(\beta\)), and energy retention.
\end{itemize}

\section{Experimental Setup}
\subsection{Hardware Implementation}
The experiment will be conducted using a superconducting transmon qubit coupled to a microwave cavity. The system will be housed inside a dilution refrigerator (10-15mK) to minimize thermal noise. The setup consists of:
\begin{itemize}
    \item \textbf{Superconducting qubits} (IBM-Q or Rigetti hardware)
    \item \textbf{Microwave cavity resonators} for Jaynes-Cummings interaction
    \item \textbf{Classical control electronics} (FPGA-based feedback implementation)
    \item \textbf{Measurement setup} for state tomography
\end{itemize}

\subsection{System Hamiltonian}
The system Hamiltonian follows the Jaynes-Cummings model:
\begin{equation}
H = \frac{\hbar \omega_q}{2} \sigma_z + \hbar \omega_c a^\dagger a + \hbar g (\sigma_+ a + \sigma_- a^\dagger)
\end{equation}
where:
\begin{itemize}
    \item \( \omega_q \) and \( \omega_c \) are qubit and cavity frequencies.
    \item \( g \) is the coupling strength.
    \item \( \sigma_+ \), \( \sigma_- \) are the qubit raising/lowering operators.
    \item \( a^\dagger \), \( a \) are the cavity photon creation/annihilation operators.
\end{itemize}

\textbf{Feedback is introduced} with a control Hamiltonian:
\begin{equation}
H_{FB} = \beta ( \sigma_x \otimes \sigma_x + (a + a^\dagger) )
\end{equation}
where \( \beta \) is an adaptive parameter based on system concurrence.

\subsection{Feedback Mechanism}
The system will be initialized in a Bell state, and a feedback loop will be applied after a tunable delay \( \tau_f \). The feedback strength \( \beta \) is dynamically adjusted based on concurrence:
\begin{equation}
F(\rho) = (1 - C) e^{-C}
\end{equation}
where \( C \) is the concurrence of the two-qubit reduced density matrix.

\section{Experimental Procedure}

\subsection{Step 1: System Initialization}
\begin{itemize}
    \item Prepare the qubit-cavity system in a Bell state.
    \item Use a microwave drive to simulate Rabi oscillations.
\end{itemize}

\subsection{Step 2: Feedback Application}
\begin{itemize}
    \item Apply feedback control with varying delay times (\(\tau_f\)).
    \item Use FPGA-based real-time measurement to estimate system concurrence.
    \item Dynamically adjust feedback strength (\(\beta\)) based on concurrence.
\end{itemize}

\subsection{Step 3: Measurement and Data Collection}
\begin{itemize}
    \item Perform Quantum State Tomography (QST) to reconstruct the system's density matrix.
    \item Compute:
    \begin{itemize}
        \item Concurrence \( C \) (Entanglement measure)
        \item Ergotropy \( W_{ex} \) (Extractable Work)
        \item Cavity photon number \( \langle n \rangle \)
    \end{itemize}
    \item Analyze how feedback strength and delay impact coherence and work extraction.
\end{itemize}

\section{Expected Results}
\begin{itemize}
    \item Delayed feedback should enhance work retention compared to uncontrolled evolution.
    \item Ergotropy and concurrence may not be strictly correlated, indicating coherence-based correlations play a role.
    \item An optimal feedback delay (\(\tau_f\)) exists that maximizes extractable work.
    \item Superconducting qubit data should match numerical simulations from \texttt{ah.py}.
\end{itemize}


\section{Conclusion and Future Work}
This study could validate feedback-controlled quantum work extraction using superconducting qubits. Future directions include:
\begin{itemize}
    \item Testing SER feedback on larger qubit arrays.
    \item Exploring alternative feedback functions based on coherence instead of concurrence.
    \item Real-world applications in quantum heat engines and quantum batteries.
\end{itemize}

\section{References}
\begin{enumerate}
    \item Jaynes, E.T., \& Cummings, F.W. (1963). \textit{Comparison of quantum and semiclassical radiation theories with application to the beam maser}. IEEE Transactions on Information Theory.
    \item Wallace, R. (2025). \textit{Structured Energy Return in Quantum Systems: Extended Analysis}. arXiv preprint.
\end{enumerate}

\end{document}
